\documentclass[11pt,a4paper]{article} 

\usepackage[a4paper,margin=2cm]{geometry}
\usepackage[T1]{fontenc}
%\usepackage{pslatex}
\usepackage[utf8]{inputenc}
\usepackage[french]{babel} 
\usepackage{graphicx} 
\usepackage{amsmath} 
\setlength{\unitlength}{1mm}
\usepackage{enumitem}
\usepackage{cancel}
\usepackage{amssymb} % pour les ensembles NN
\usepackage{mathrsfs} % pour les hypothèses de récurrences \PP
\usepackage{fancyhdr}
\usepackage{fancybox}
\usepackage{titling}
\usepackage[explicit]{titlesec}
\usepackage{mathtools}
\usepackage{eurosym}
\usepackage{stmaryrd} % pour parallèle // \sslash
\usepackage{physics}
\usepackage{pgfplots}
\usepackage[lined,boxed,commentsnumbered, ruled,vlined,linesnumbered, french, onelanguage]{algorithm2e}
\usepackage{hyperref}

\setlength{\headheight}{13.59999pt}

\pgfplotsset{compat=1.15}

\definecolor{Honeydew1}{rgb}{.94,1,.94}
\definecolor{lavendermist}{rgb}{0.9, 0.9, 0.98}

\newcommand{\res}{\colorbox{lavendermist}}
\newcommand{\resm}[1]{\colorbox{lavendermist}{$\displaystyle #1$}}

% vecteurs
\newcommand*\colvec[3][]{
    \begin{pmatrix}\ifx\relax#1\relax\else#1\\\fi#2\\#3\end{pmatrix}
}

\pagestyle{fancy}

% ensembles N,Z,Q,D,R,C
\DeclareMathOperator{\NN}{\mathbb{N}}
\DeclareMathOperator{\ZZ}{\mathbb{Z}}
\DeclareMathOperator{\QQ}{\mathbb{Q}}
\DeclareMathOperator{\DD}{\mathbb{D}}
\DeclareMathOperator{\RR}{\mathbb{R}}
\DeclareMathOperator{\CC}{\mathbb{C}}
\DeclareMathOperator{\e}{e}

% droite D et plan P
\DeclareMathOperator{\GP}{\mathscr{P}}
\DeclareMathOperator{\GD}{\mathscr{D}}
\DeclareMathOperator{\GA}{\mathscr{A}}
\DeclareMathOperator{\GC}{\mathscr{C}}
\DeclareMathOperator{\GS}{\mathscr{S}}

\newcommand*{\QEDA}{\null\nobreak\hfill\ensuremath{\blacksquare}}
\newcommand*{\QEDB}{\null\nobreak\hfill\ensuremath{\square}}
\newcommand{\usd}{\frac{1}{2}}
\newcommand{\tsd}{\frac{3}{2}}
\newcommand{\csd}{\frac{5}{2}}
\newcommand{\usq}{\frac{1}{4}}
\newcommand{\tsq}{\frac{3}{4}}
\newcommand{\csq}{\frac{5}{4}}
\newcommand{\ssq}{\frac{6}{4}}
\newcommand{\ush}{\frac{1}{8}}
\newcommand{\psd}{\frac{\pi}{2}}

% angle deux vecteurs
\newcommand{\angv}[2]{(\widehat{\overrightarrow{#1},\overrightarrow{#2}})}
\newcommand{\vf}[1]{\overrightarrow{#1}}

% conjugué complexe
\newcommand{\zb}[1]{\overline{#1}}

\newcommand{\monNom}{Marilou Bernard de Courville}
\newcommand{\maClasse}{MP$^{*}$, Charlemagne}


\lhead{\monNom}
\chead{}
\rhead{\maClasse}

\lfoot{\jobname.tex}
\cfoot{}
\rfoot{\thepage}
\renewcommand{\headrulewidth}{0.4pt}
\renewcommand{\footrulewidth}{0.4pt}

\setlength{\droptitle}{-1.5cm}

\makeatletter
\renewcommand{\maketitle}{
  \thispagestyle{empty}
  \begin{center}
  \shadowbox{\parbox{5in}{%
     \centering%
     \textrm{\textbf{\Large \@title}}\\
     \vspace{0.2cm}
     \textrm{\large \@author}\\
     \vspace{0.2cm}
     \textrm{\large \@date}
  }} 
  \end{center}
  \null
}

\author{\monNom, \maClasse}
\title{Mise en Cohérence des Objectifs du TIPE}
\date{\today} 


%\begin{enumerate}[label=(\alph*)]
%	\item	
%\end{enumerate}

%%%%%%%%%%%%%%%%%%%%%%%%%%%%%%%%%%%%%%%%%%%%%%%%%%%%%%%%%%%%%%%%%%%%%%%%%%
% NOTES


\begin{document}

\maketitle

\tableofcontents

\thispagestyle{fancy}

\section{Titre - 20 mots}

\begin{center}
    Conception d'une intelligence artificielle capable de jouer de manière autonome et performante au jeu Snake à l'aide d'un algorithme génétique  
\end{center}

\textit{20 mots}

\section{Motivation pour le choix du sujet - 50 mots}

%+ Ce qui te motive!

Le jeu de snake, ayant émergé dans les années 80 en salle d'arcade,
permet l'application d'un algorithme génétique.
Ce genre d'algorithme s'inspirant de l'évolution naturelle a de 
nombreuses applications dans le domaine de l'ingénierie, permettant de concevoir 
des pièces de forme optimale jusqu'à permettre de créer des tests d'applications
efficaces.


Ma passion pour l'informatique et les jeux vidéo, combinée à l'importance 
croissante de l'intelligence artificielle, m'a motivée à explorer les algorithmes 
génétiques et les réseaux neuronaux dans le context du jeu Snake.
Je souhaite comprendre les fondements théoriques de cette technologie et son 
application pratique dans un contexte ludique.

%passion de l'informatique combiné à mon intérêt pour les jeux vidéos
%mettre en pratique tout en comprenant les fondements théoriques
%AI problématique actuelle
%première étape pour mettre au service de domaines d'applications au delà de l'informatique

\textit{50 - 49 mots}

\section{Lien avec le thème de l'année (Jeux et Sports) - 50 mots}

L'engouement persistant pour le jeu Snake lui confère un caractère ludique 
indéniable. En concevant une IA performante et autonome pour résoudre ce jeu 
d'arcade des années 80, nous établissons un lien évident avec le thème de 
l'année tout en explorant les fondements théoriques des réseaux neuronaux et 
des algorithmes génétiques.

\textit{50 mots}

\section{Bibliographie commentée - 650 mots}

% \subsection{Introduction de snake}

Le jeu Snake est né en 1976 en tant que borne d'arcade fabriquée par l'entreprise Gremlin,
 initialement sous le nom de Blockade. Il a été conçu par Lane Hauck, Ago Kiss et Bob Pecarero. [1]

%[1]  https://allincolorforaquarter.blogspot.com/2015/09/the-ultimate-so-far-history-of-gremlin_25.html The Golden Age Arcade Historian "The Ultimate (So-Far) History of Gremlin Industries Part 2"

L'objectif du jeu Snake est de contrôler un serpent qui se déplace à travers un espace limité, 
en mangeant des objets pour grandir. Le joueur doit éviter de se heurter aux murs ou de se mordre 
la queue, car cela entraînerait la fin du jeu. L'objectif ultime est de survivre le plus longtemps 
possible et de marquer le plus de points en faisant grandir le serpent.

L'idée est de concevoir et d'entraîner une IA pour jouer de manière autonome à ce jeu en se 
basant sur un algorithme génétique appliqué à un réseau de neurones.

% \subsection{Introduction réseaux neuronaux}


% [2] https://www.ibm.com/fr-fr/topics/neural-networks Qu'est-ce qu'un réseau de neurones ?


En 1943, Warren McCulloch et Walter Pitts comparent des neurones à la logique booléenne. [2]
L'idée est de supposer qu'un neurone est activé à partir du moment où il a un seuil suffisant,
et l'activation ou non de différents neurones ont un effet sur les décisions prises par l'ensemble
de neurones, donc le réseau neuronal.
On peut donc aisément trouver une application de ce modèle au jeu de Snake: on veut en effet 
modéliser un serpent, ainsi, on utilise un réseau neuronal pour modéliser le cerveau du serpent.
C'est de cette façon dont notre serpent va décider de quelle façon il va se déplacer. 

Frank Rosenblatt, en 1958, a introduit la notion de poids [2], qui permet d'accorder plus ou moins d'importance à certains neurones, donc à certaines informations sur la situation dans laquelle le serpent se trouve.

C'est donc une amélioration du modèle de McCulloch et Pitts, qui permet de rendre le réseau neuronal plus performant.
Il faut donc trouver une façon de déterminer les poids optimaux, c'est-à-dire les poids qui permettent au serpent de survivre le plus longtemps possible.
Pour cela, on va faire des opérations successives sur le réseau neuronal.
Au lieu d'étudier un seul serpent, on va donc étudier une population de serpents, et on va faire évoluer cette population de serpents.
On va donc appliquer différents opérateurs génétiques sur cette population, qui vont permettre de faire converger les poids des différents neurones vers les poids optimaux. [3][4]

% \subsection{Introduction algorithmes génétiques}

% [3] http://www0.cs.ucl.ac.uk/staff/W.Langdon/ftp/papers/poli08_fieldguide.pdf A Field Guide to Genetic Programming, Riccardo Poli; William B. Langdon; Nicholas F. McPhee

% [4] A Genetic Algorithm Tutorial, Darrell Whitley

Cet ensemble d'opérateurs génétiques sont effectués dans un type d'algorithme évolutionniste, appelé algorithme génétique, développés par John Holland en 1975.
Le but est ainsi d'imiter l'évolution naturelle, en sélectionnant des serpents performant et en les faisant se reproduire, créant ainsi des enfants dont le réseau neuronal est composé d'une partie du réseau neuronal de chaque parent.
De plus, on fait subir au réseau neuronal des enfants une mutation, qui permet d'engendrer des possibles améliorations du réseau neuronal.

Ce qui est délicat et qui peut être décisif dans l'efficacité de l'algorithme, est la façon dont on sélectionne les serpents qui vont se reproduire.
Cette sélection est effectuée à l'aide d'une fonction de fitness, qui permet de déterminer la performance d'un serpent.
Cette fonction dépend ainsi de l'âge du serpent et du nombre de pommes mangées par le serpent, mais on a la liberté de choisir la dépendance de cette fonction en ces paramètres.
Il est ainsi intéressant d'étudier l'effet de différentes fonctions de fitness sur l'efficacité de l'algorithme génétique.

% \subsection{Intérêt pour des questions théoriques: comment modéliser le sys. pour prouver les résultats?}

On peut enfin se demander comment formaliser ces différents concepts, et comment au moins comprendre pourquoi l'algorithme génétique fonctionne.
On s'intéresse alors d'une part aux graphes, pour comprendre la structure du réseau neuronal, et d'autre part aux schemata de Holland, qui permettent de comprendre comment les opérateurs génétiques agissent sur la population de serpents.

% [4] A Genetic Algorithm Tutorial, Darrell Whitley

% [5] Handbook Of Genetic Algorithms, L. Davis

\textit{461 mots}

\section{Problématique retenue - 50 mots}

Notre sujet vise à répondre aux questions suivantes:
Comment entraîner une IA pour être compétitive au jeu Snake à l'aide d'un algorithme génétique?
Quelles optimisations et paramètres sont à prendre en compte pour obtenir un bon score.
Comment garantir l'efficacité du réseau de neurones obtenu?

\textit{45 mots}

\section{Objectifs du TIPE - 100 mots}

L'objectif proposé est la mise en place d'une intelligence artificielle pouvant jouer à Snake.

D'abord, on cherche à coder un jeu de Snake fonctionnel: on se servira de la bibliothèque PyGame de Python.

Ensuite, on créé un réseau neuronal: on codera les fonctions permettant la vision du serpent, l'activation des neurones et leur réponse, mais également des fonctions permettant l'exploitation du réseau neuronal par l'algorithme génétique.

En effet, notre dernier objectif est de coder des fonctions permettant d'appliquer des opérateurs génétiques sur le réseau neuronal.

On pourra alors se pencher sur des aspects théoriques de cette méthode.

-->

L'objectif est de concevoir une intelligence artificielle capable de jouer de manière autonome et performante au jeu Snake.
Pour cela un réseau de neurones est entraîné par un algorithme génétique qui simule la sélection naturelle.
Les entrées du réseau de neurones sont constituées de paramètres de vision obtenus à partir de la tête du serpent.
Plusieurs fonctions de fitness sont considérées afin d'obtenir le meilleur score et d'éviter les boucles.
Le tout est codé en python et seule la librairie PyGame est utilisée et toutes les fonctions d'algorithme génétique ou réseau de neurones sont ré-implantées.
Les aspects théoriques et de convergence des stratégies proposées sont étudiées.

\textit{97 mots}

\section{Mots-clés - 5 en français et en anglais}

\footnotesize
\begin{tabular}{||r|c|c|c|c|c||} \hline
	Français & Intelligence artificielle & Algorithme génétique & Snake & Réseaux neuronaux & Crossing-over \\ \hline \hline
	Anglais & Artificial Intelligence & Genetic Algorithm & Snake & Neural Networks & Crossover \\
		\hline
	\end{tabular}

\section{Bibliographie}

[1] \href{https://allincolorforaquarter.blogspot.com/2015/09/the-ultimate-so-far-history-of-gremlin_25.html}{The Golden Age Arcade Historian "The Ultimate (So-Far) History of Gremlin Industries Part 2"}

[2] \href{https://www.ibm.com/fr-fr/topics/neural-networks}{Qu'est-ce qu'un réseau de neurones ?}

[3] \href{http://www0.cs.ucl.ac.uk/staff/W.Langdon/ftp/papers/poli08_fieldguide.pdf}{A Field Guide to Genetic Programming, Riccardo Poli; William B. Langdon; Nicholas F. McPhee}

[4] A Genetic Algorithm Tutorial, Darrell Whitley

[5] Handbook Of Genetic Algorithms, L. Davis

\end{document}





Développement d'une méthode de prise de décision : L'agent du jeu Snake prend une décision pour chaque direction à chaque image du jeu (haut, bas, gauche, droite) en utilisant un réseau de neurones avec deux couches cachées.

Définition des entrées du réseau de neurones: Les entrées comprennent la distance de Manhattan de la tête du serpent au fruit dans chaque direction, la quantité d'espace disponible si le serpent se déplace dans chaque direction (calculée par une recherche en largeur), et la longueur du serpent.


\documentclass[10pt]{beamer}
\usetheme{Goettingen}

\usepackage[french]{babel}
\usepackage[T1]{fontenc}

\usepackage{amsmath}
\usepackage{algpseudocode}

\usepackage{algorithm}
%\usepackage{algorithmic}
\newcommand{\kw}[1]{\textrm{#1}}

\usepackage{forest}

%\usepackage{scrextend}
%\changefontsizes{7.5pt}

\usefonttheme[onlymath]{serif}

\usepackage{xcolor}
\usepackage{graphicx} 
\usepackage{amsmath} 
\setlength{\unitlength}{1mm}
%\usepackage{enumitem}
\usepackage{cancel}
\usepackage{amssymb} % pour les ensembles NN
\usepackage{mathrsfs} % pour les hypothèses de récurrences \PP
%\usepackage{fancyhdr}
\usepackage{multicol}


\usepackage{booktabs}
\usepackage{tabularx}
\usepackage{array,collcell}
\newcommand\AddLabel[1]{%
  \refstepcounter{equation}% increment equation counter
  (\theequation)% print equation number
  \label{#1}% give the equation a \label
}
\newcolumntype{M}{>{\hfil}X<{\hfil}} % mathematics column
%\newcolumntype{M}{>{\hfil$\displaystyle}X<{$\hfil}} % mathematics column
%\newcolumntype{L}{>{\collectcell\AddLabel}r<{\endcollectcell}}
\renewcommand\tabularxcolumn[1]{m{#1}}% for vertical centering text in X column

\usepackage{tikz}
\usetikzlibrary{calc, shapes}

\usepackage{tkz-graph}
\tikzset{EdgeStyle/.append style = {->}}
\tikzset{LabelStyle/.style= {draw,
fill = white,
text = black}}

\newcommand*\circled[1]{\tikz[baseline=(char.base)]{
            \node[shape=circle,draw,inner sep=1pt] (char) {#1};}}

% ensembles N,Z,Q,D,R,C
\DeclareMathOperator{\NN}{\mathbb{N}}
\DeclareMathOperator{\ZZ}{\mathbb{Z}}
\DeclareMathOperator{\QQ}{\mathbb{Q}}
\DeclareMathOperator{\DD}{\mathbb{D}}
\DeclareMathOperator{\RR}{\mathbb{R}}
\DeclareMathOperator{\CC}{\mathbb{C}}
\DeclareMathOperator*{\argmax}{\arg\!\max}

\title{TIPE 2024}
\subtitle{Apprendre à une intelligence artificielle à jouer à Snake en utilisant un algorithme génétique}
\author{Marilou Bernard de Courville}
\institute{Lycée Charlemagne}
\date{\today}

\setbeamertemplate{footline}[frame number]


\begin{document}
 
\begin{frame}
    \titlepage
\end{frame}

%\begin{frame}
%    \frametitle{Table des matières}
%    \tableofcontents
%\end{frame}

\section{Introduction}

\begin{frame}
\frametitle{Introduction}
\framesubtitle{Problématique et pertinence au regard du thème de l'année}

\begin{itemize}

\item \textbf{Le jeu de Snake:} piloter un serpent dans le but de
manger des pommes, sans rentrer dans les murs ni se replier sur 
soi-même.

\item \textbf{Objectif:} mettre en place une intelligence
artificielle pouvant jouer au jeu de Snake.

\item \textbf{Le moyen d'y parvenir:} utiliser un algorithme génétique,
qui s'inspire de l'évolution naturelle.

\item 

\end{itemize}

\end{frame}

\section{\textbf{1\ier{} objectif}: un trajet efficace en métro}

\subsection{Données du problème}

\subsection{Algorithme de Dijkstra, structures de données}

\begin{frame}
\frametitle{Trajet le plus court en métro - modélisation}
\framesubtitle{Modèles de données utilisés (types)}

\small
\begin{itemize}
\item Un tas est une représentation organisée en arbre des stations en fonction d'une priorité (distance du chemin).
\item Représentation en tas (heap) efficace en complexité pour accéder au n\oe ud de priorité minimum et mettre à jour les priorités.
\item Trois structures étudiées: tas non modifiables (\textit{immutable heaps}), tas modifiables, et tarbres (\textit{treap}).
\item Programmes réalisés en OCaml utilisent seulement le module \texttt{Hashtbl} de la bibliothèque standard pour manipuler les tas.
\end{itemize}

\begin{columns}[T]
\begin{column}{0.5\textwidth}
\tiny
\forestset{default preamble={for tree={circle,draw}}}
\begin{forest}
[2[4[5[14][9]][6[11][10]]][7[8[12]][13]]]
\end{forest}

\begin{tabular}{|c|c|c|c|c|c|c|c|c|c|c|c|c|}
    \hline
    2 & 4 & 7 & 5 & 6 & 8 & 13 & 14 & 9 & 11 & 10 & 12\\
    \hline
\end{tabular}
\\
\begin{tabular}{|c|c|c|c|c|c|c|c|c|c|c|c|c|}
\hline
\textbf{Élément} & 2 & 4 & 5 & 6 & 7 & 8 & 9 & 10 & 11 & 12 & 13 & 14 \\
\hline
\textbf{Position} & 0 & 1 & 3 & 4 & 2 & 5 & 8 & 10 & 9 & 11 & 6 & 7 \\
\hline
\end{tabular}
\end{column}
\begin{column}{0.5\textwidth}
\tiny
\forestset{default preamble={for tree={circle,draw}}}
rg0: 
\begin{forest}
[]
\end{forest}
rg1: 
\begin{forest}
[[]]
\end{forest}
rg2: 
\begin{forest}
[[[]][]]
\end{forest}
rg3: 
\begin{forest}
[[[[]][]][[]][]]
\end{forest}
\end{column}
\end{columns}
\end{frame}


\begin{frame}
\frametitle{Trajet le plus court en métro - algorithme}
\framesubtitle{Algorithme de Dijkstra et influence des structures de données sur la complexité}

\begin{columns}[T]
\begin{column}{0.5\textwidth}
\scriptsize
\begin{algorithmic}[0]
\Require Un graphe $G=(V,A)$, $V$ sommets, $A$ arêtes 
\Require Un noeud source $s$
\Ensure $d$ tableau des plus court chemins de $s$ vers $v\in V$
\ForAll {$v \in V[G]$}
    \State $d[v] \leftarrow +\infty$, $\kw{père}[v] \leftarrow \kw{None}$
\EndFor
\State $d[s] \leftarrow 0$, $S \leftarrow \kw{$\emptyset$}$, $Q \leftarrow V[G]$
\While{$Q \neq \emptyset$} 
    \State $u \leftarrow  \kw{Extrait}_{\kw{Min}}(Q)$
    \State $S \leftarrow  S \cup \{u\}$
    \ForAll{arête $(u,v)$ d'origine $u$}  
    \If{$d[u] + w(u,v) < d[v]$}
        \State $d[v] \leftarrow d[u] + w(u,v)$, $\kw{père}[v] := u$ 
    \EndIf
    \EndFor
\EndWhile
\end{algorithmic}
\end{column}
\begin{column}{0.5\textwidth}
\scriptsize

\begin{table}[h]\vspace{-0.5cm}
\centering
\caption{Dijkstra: complexité (opérations)}
%\label{tab:dijkstra-complexity}
\hspace{-0.5cm}
\begin{tabular}{|l|l|}
\hline
\textbf{Implantation}	& \textbf{Complexity} \\ \hline
Naif					& $\mathcal{O}\left(V^2 + A\right)$             \\ \hline
Tas						& $\mathcal{O}\left((V + A) \log V\right)$      \\ \hline
\end{tabular}
\end{table}

\end{column}
\end{columns}

\end{frame}

\subsection{Résultat: stations les plus éloignées}

\section{\textbf{2\ieme{} objectif}: un trajet constitué de plusieurs types de chemins}

\subsection{Cas préliminaire du métro}

\begin{frame}
\frametitle{Plus compliqué: plus petit chemin passant par toutes les lignes du métro}
\framesubtitle{Étude de la résolution à l'aide d'un solveur linéaire}
% https://interstices.info/quel-trajet-optimal-pour-passer-au-moins-une-fois-par-toutes-les-lignes-de-metro/
\begin{itemize}
\item Sujet traité par Florian Sikora $[7]$ par étude de graphe coloré pour le réseau du métro;
\item Sommet: station, arête: trajet entre deux stations connexes, arête colorée par la couleur de la ligne reliant les stations
\item Problème du "Generalised directed rural postman" $[2]$;
\item Problème NP-difficile: pas de solution en temps polynomial $[6]$;
\item Résolution requiert l'utilisation d’un solveur linéaire (CPLEX d'IBM) pour "integer linear programming" (ILP) $[9]$;
\item Fait intervenir une matrice de contraintes (MIP) de \texttt{1270x1847, 6999 coeffs non nuls}.
\end{itemize}
\end{frame}

\subsection{Formalisation du problème}

\begin{frame}
\frametitle{Formalisation du problème: variables}
%\framesubtitle{Objectif - caractérisation des variables}

\begin{itemize}
\item Ensemble $V$ des sommets, $A$ des arêtes, $C$ des couleurs.
$(u,v)\in V^2$, $u \xrightarrow[l\in C]{} v\in A$.

\item $x_{u,v,l}\in \lbrace 0,1 \rbrace$: variable binaire pour chaque arc $u \xrightarrow[l]{} v$ (sur ligne $l$), avec $x=1$ si l'arc est considéré, $x=0$ sinon.

\item $w_{u,v,l}\in \NN$: est le temps pour parcourir l'arête $u \xrightarrow[l]{} v$

\item $(u,v)\in V^2, f_{u,v,l}, y_v \in\NN$ sont les flots des arcs/sommets: positifs si l'arc/sommet est sur le chemin considéré.

\item $s, t$ sont les points de départ/arrivée fictifs (temps nul pour rejoindre tout sommet).
\item  $\forall ((u,v,l_1), (v,w,l_2)) \in A^2, z_{u,v,w,l_1,l_2} \in \lbrace 0,1 \rbrace$ indique si deux arêtes sont utilisées consécutivement $u \xrightarrow[l_1]{} v\xrightarrow[l_2]{} w$.
\end{itemize}
\end{frame}


\subsection{Résolution, application au cadre de la ville}



\section{Conclusion}

\begin{frame}
\frametitle{Conclusion}
\begin{itemize}
\item Deux problématiques urbaines traitées: 
  \begin{itemize}
  \item optimisation d'un trajet en métro
  \item parcours touristique efficace d'une ville en empruntant différents types de chemins
  \end{itemize}
\item Pertinence de la modélisation des problèmes urbains par des graphes pour les résoudre.
\item Application de la recherche opérationnelle pour trouver une solution.
  \begin{itemize}
  \item Optimisation fait intervenir un grand nombre de contraintes
résultant en des problèmes combinatoires complexes sans solution analytique.
  \item Importance du choix des algorithmes et structures de données pour obtenir des solutions pratiques efficaces.
  \end{itemize}
\end{itemize}
\end{frame}


\end{document}
\documentclass[11pt,a4paper]{article} 

\usepackage[a4paper,margin=2cm]{geometry}
\usepackage[T1]{fontenc}
%\usepackage{pslatex}
\usepackage[utf8]{inputenc}
\usepackage[francais]{babel} 
\usepackage{graphicx} 
\usepackage{amsmath} 
\setlength{\unitlength}{1mm}
\usepackage{enumitem}
\usepackage{cancel}
\usepackage{amssymb} % pour les ensembles NN
\usepackage{mathrsfs} % pour les hypothèses de récurrences \PP
\usepackage{fancyhdr}
\usepackage{tikz,tkz-tab,tikz-3dplot,tkz-euclide}
\usetkzobj{all}
\usetikzlibrary{angles,quotes,intersections,calc,through}
\usepackage{fancybox}
\usepackage{titling}
\usepackage[explicit]{titlesec}
\usepackage{mathtools}
\usepackage{eurosym}
\usepackage{stmaryrd} % pour parallèle // \sslash
\usepackage{physics}
\usepackage{pgfplots}
\usepackage[lined,boxed,commentsnumbered, ruled,vlined,linesnumbered, french, onelanguage]{algorithm2e}
\usepackage{hyperref}



\definecolor{Honeydew1}{rgb}{.94,1,.94}
\definecolor{lavendermist}{rgb}{0.9, 0.9, 0.98}

\newcommand{\res}{\colorbox{lavendermist}}
\newcommand{\resm}[1]{\colorbox{lavendermist}{$\displaystyle #1$}}

% vecteurs
\newcommand*\colvec[3][]{
    \begin{pmatrix}\ifx\relax#1\relax\else#1\\\fi#2\\#3\end{pmatrix}
}

\pagestyle{fancy}

% ensembles N,Z,Q,D,R,C
\DeclareMathOperator{\NN}{\mathbb{N}}
\DeclareMathOperator{\ZZ}{\mathbb{Z}}
\DeclareMathOperator{\QQ}{\mathbb{Q}}
\DeclareMathOperator{\DD}{\mathbb{D}}
\DeclareMathOperator{\RR}{\mathbb{R}}
\DeclareMathOperator{\CC}{\mathbb{C}}
\DeclareMathOperator{\e}{e}

% droite D et plan P
\DeclareMathOperator{\GP}{\mathscr{P}}
\DeclareMathOperator{\GD}{\mathscr{D}}
\DeclareMathOperator{\GA}{\mathscr{A}}
\DeclareMathOperator{\GC}{\mathscr{C}}
\DeclareMathOperator{\GS}{\mathscr{S}}

\newcommand*{\QEDA}{\null\nobreak\hfill\ensuremath{\blacksquare}}
\newcommand*{\QEDB}{\null\nobreak\hfill\ensuremath{\square}}
\newcommand{\usd}{\frac{1}{2}}
\newcommand{\tsd}{\frac{3}{2}}
\newcommand{\csd}{\frac{5}{2}}
\newcommand{\usq}{\frac{1}{4}}
\newcommand{\tsq}{\frac{3}{4}}
\newcommand{\csq}{\frac{5}{4}}
\newcommand{\ssq}{\frac{6}{4}}
\newcommand{\ush}{\frac{1}{8}}
\newcommand{\psd}{\frac{\pi}{2}}

% angle deux vecteurs
\newcommand{\angv}[2]{(\widehat{\overrightarrow{#1},\overrightarrow{#2}})}
\newcommand{\vf}[1]{\overrightarrow{#1}}

% conjugué complexe
\newcommand{\zb}[1]{\overline{#1}}

\newcommand{\monNom}{Marilou Bernard de Courville}
\newcommand{\maClasse}{MP$^{*}$, Charlemagne}


\lhead{\monNom}
\chead{}
\rhead{\maClasse}

\lfoot{\jobname.tex}
\cfoot{}
\rfoot{\thepage}
\renewcommand{\headrulewidth}{0.4pt}
\renewcommand{\footrulewidth}{0.4pt}

\setlength{\droptitle}{-1.5cm}

\makeatletter
\renewcommand{\maketitle}{
  \thispagestyle{empty}
  \begin{center}
  \shadowbox{\parbox{5in}{%
     \centering%
     \textrm{\textbf{\Large \@title}}\\
     \vspace{0.2cm}
     \textrm{\large \@author}\\
     \vspace{0.2cm}
     \textrm{\large \@date}
  }} 
  \end{center}
  \null
}

\author{\monNom, \maClasse}
\title{Mise en Cohérence des Objectifs du TIPE}
\date{\today} 


%\begin{enumerate}[label=(\alph*)]
%	\item	
%\end{enumerate}

%%%%%%%%%%%%%%%%%%%%%%%%%%%%%%%%%%%%%%%%%%%%%%%%%%%%%%%%%%%%%%%%%%%%%%%%%%
% NOTES


\begin{document}

\maketitle

\tableofcontents

\thispagestyle{fancy}

\section{Titre - 20 mots}

\begin{center}
    Apprendre à une intelligence artificielle à jouer à Snake en utilisant un algorithme génétique.
\end{center}

\textit{14 mots}

\section{Motivation pour le choix du sujet - 50 mots}

Le jeu de snake, jeu ayant émergé dans les années 80 en salle d'arcade, permet d'illustrer le principe d'algorithme évolutionniste, plus précisément d'algorithme génétique.
Ce genre d'algorithme s'inspirant de l'évolution naturelle, il a ensuite de nombreuses applications dans le domaine de l'ingéniérie, le rendant de plus en plus pertinant.

\textit{49 mots}

\section{Lien avec le thème de l'année (Jeux et Sports) - 50 mots}

\section{Bibliographie commentée - 650 mots}



\section{Problématique - 50 mots}

La problématique est comment rendre une intelligence artificielle compétitive au jeu Snake, à l'aide de différents algorithmes d'apprentissage notamment des algorithmes génétiques.

\textit{25 mots}

\section{Objectifs - 100 mots}


\section{Mots-clés - 5 en français et en anglais}

\begin{tabular}{||r|c|c|c||} \hline
	Français & Intelligence artificielle & Algorithme génétique & Snake & & \\ \hline \hline
	Anglais & Artificial Intelligence & Genetic algorithm & Snake & & \\
		\hline
	\end{tabular}

\section{Bibliographie}

















\end{document}
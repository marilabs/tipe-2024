\documentclass[10pt]{beamer}
\usetheme{Goettingen}



\usepackage[french]{babel}
\usepackage[T1]{fontenc}

\usepackage{amsmath}
\usepackage{algpseudocode}

\usepackage{algorithm}
%\usepackage{algorithmic}
\newcommand{\kw}[1]{\textrm{#1}}

\usepackage{forest}

%\usepackage{scrextend}
%\changefontsizes{7.5pt}

\usefonttheme[onlymath]{serif}

\usepackage{xcolor}
\usepackage{graphicx} 
\usepackage{amsmath} 
\setlength{\unitlength}{1mm}
%\usepackage{enumitem}
\usepackage{cancel}
\usepackage{amssymb} % pour les ensembles NN
\usepackage{mathrsfs} % pour les hypothèses de récurrences \PP
%\usepackage{fancyhdr}
\usepackage{multicol}


\usepackage{booktabs}
\usepackage{tabularx}
\usepackage{array,collcell}
\newcommand\AddLabel[1]{%
  \refstepcounter{equation}% increment equation counter
  (\theequation)% print equation number
  \label{#1}% give the equation a \label
}
\newcolumntype{M}{>{\hfil}X<{\hfil}} % mathematics column
%\newcolumntype{M}{>{\hfil$\displaystyle}X<{$\hfil}} % mathematics column
%\newcolumntype{L}{>{\collectcell\AddLabel}r<{\endcollectcell}}
\renewcommand\tabularxcolumn[1]{m{#1}}% for vertical centering text in X column

\usepackage{tikz}
\usetikzlibrary{calc, shapes}

\usepackage{tkz-graph}
\tikzset{EdgeStyle/.append style = {->}}
\tikzset{LabelStyle/.style= {draw,
fill = white,
text = black}}

\newcommand*\circled[1]{\tikz[baseline=(char.base)]{
            \node[shape=circle,draw,inner sep=1pt] (char) {#1};}}

% ensembles N,Z,Q,D,R,C
\DeclareMathOperator{\NN}{\mathbb{N}}
\DeclareMathOperator{\ZZ}{\mathbb{Z}}
\DeclareMathOperator{\QQ}{\mathbb{Q}}
\DeclareMathOperator{\DD}{\mathbb{D}}
\DeclareMathOperator{\RR}{\mathbb{R}}
\DeclareMathOperator{\CC}{\mathbb{C}}
\DeclareMathOperator*{\argmax}{\arg\!\max}

\title{TIPE 2024}
\subtitle{Apprendre à une intelligence artificielle à jouer à Snake en utilisant un algorithme génétique}
\author{Marilou Bernard de Courville}
\institute{Lycée Charlemagne}
\date{\today}

\setbeamertemplate{footline}[frame number]


\begin{document}
 
\begin{frame}
    \titlepage
\end{frame}

%\begin{frame}
%    \frametitle{Table des matières}
%    \tableofcontents
%\end{frame}

\section{Introduction}

\begin{frame}
\frametitle{Introduction}
\framesubtitle{Problématique et pertinence au regard du thème de l'année}

\begin{itemize}

\item \textbf{Le jeu de Snake:} piloter un serpent sur une grille dans le but de
manger des pommes, sans rentrer dans les murs ni se replier sur 
soi-même.

\item \textbf{Objectif:} mettre en place une intelligence
artificielle pouvant jouer au jeu de Snake, apprenant de manière autonome.

\item \textbf{Le moyen d'y parvenir:} utiliser un algorithme génétique,
qui s'inspire de l'évolution naturelle.

\end{itemize}

\end{frame}

\section{Le principe}

\begin{frame}
\frametitle{Le principe}
\end{frame}

\subsection{L'algorithme génétique}

\begin{frame}
  \frametitle{Le principe}
  \end{frame}

\subsubsection{Les réseaux neuronnaux}

\begin{frame}
  \frametitle{Le principe}
  \end{frame}

\section{L'algorithme appliqué au jeu de Snake}

\begin{frame}
  \frametitle{Le principe}
  \end{frame}

\section{La théorie}

\begin{frame}
  \frametitle{Le principe}
  \end{frame}

\subsection{L'algorithme génétique}

\begin{frame}
  \frametitle{Le principe}
  \end{frame}

\subsubsection{Les réseaux neuronnaux}

\begin{frame}
  \frametitle{Le principe}
  \end{frame}

\section{Conclusion}

\begin{frame}
  \frametitle{Le principe}
  \end{frame}

\end{document}